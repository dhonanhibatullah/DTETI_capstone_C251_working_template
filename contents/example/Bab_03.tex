

Salah satu kemampuan yang harus dimiliki oleh seorang Sarjana Teknik adalah kemampuan menganalisis sumber-sumber pustaka. Oleh karena itu, di dalam dokuman C-251 ini, mahasiswa perlu melakukan studi pustaka dan menuliskannya sampai kepada level kedalaman dan detail yang memadai. Sebelum melakukan analisis pustaka, uraikan lagi apa saja permasalahan yang ingin Anda selesaikan di dalam \textit{capstone} ini. Dari permasalahan tersebut, carilah literatur yang sudah pernah membahas dan menyelesaikan masalah yang sama, dan pilihkan pustaka kunci yang akan Anda adopsi untuk membantu menyelesaikan Anda. 

\textcolor{black}{Di bagian ini, tunjukkan bahwa masalah yang dipilih memiliki solusi terbuka (\textit{open-ended solution}), yaitu memiliki minimal 3 solusi potensial yang ada di literatur.}

\textcolor{black}{Mahasiswa juga dituntut untuk mengevaluasi solusi-solusi yang mungkin berdasarkan proses dan standar keteknikan. \textbf{Mahasiswa tidak diperkenankan hanya menyajikan solusi yang didapat di literatur. Mahasiswa perlu menunjukkan keunggulan dan kelemahan setiap solusi tersebut}. Untuk mencapai poin tersebut, di Bab ini, mahasiswa harus mengacu pada minimal 10 buah referensi dalam mengevaluasi solusi-solusi potensial dari permasalahan yang dipilih. Literatur sebaiknya diambilkan dari sumber ilmiah seperti jurnal, standard, ataupun \textit{technical report}
}

Buatlah analisis yang mendalam dan jika perlu disimulasikan metodenya untuk menyelesaikan permasalahan Anda. Uraikan secara ringkas sumber-sumber yang membahas berbagai kerangka teori dan metode-metode yang mungkin untuk digunakan secara sistematis. Gunakan \textit{heading} untuk menguraikan metode yang ada.


\section{Metode 1}
\label{sec:Metode_01}

    \lipsum[1-3]

\section{Metode 2}
\label{sec:Metode_02}

    \lipsum[4-6]
    
    \subsection{Sub-Metode 2}
    \label{subsec:Sub-Metode_02}
    
    \lipsum[7-9]

\section{Metode 3}
\label{sec:Metode_03}

    \lipsum[10-12]
    
    \subsection{Sub-Metode 3A}
    \label{subsec:Sub-Metode_03A}
    
    \lipsum[13-15]
    
    \subsection{Sub-Metode 3B}
    \label{subsec:Sub-Metode_03B}
    
    \lipsum[15-18]