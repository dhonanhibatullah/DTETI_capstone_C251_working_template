\textcolor{black}{Di bagian ini, mahasiswa diharapkan mampu mendesain dan mengalokasikan sumber daya (baik berupa manusia, fasilitas, ataupun anggaran keuangan) untuk mendukung solusi yang telah dipilih. Ini dicapai melalui:
\begin{itemize}
    \item kemampuan menguraikan solusi yang dipilih ke dalam sub-tugas (\textit{subtasks}). Hal ini didasarkan pada kenyataan bahwa untuk mencapai solusi akhir, tidak mungkin dilakukan sekali jalan. Hal yang biasa dilakukan adalah solusi akhir dipecah menjadi  beberapa sub-tugas yang dapat ditempuh baik secara serial atau paralel.
    \item pembuatan jadwal kegiatan/\textit{time table} (baik berupa \textit{Gantt chart}, \textit{timeline}, dan sejenisnya) maupun alokasi sumberdaya (seperti manusia, fasilitas, ataupun anggaran keuangan) untuk setiap sub-tugas.
\end{itemize}
Sebagai catatan, alokasi\underline{ sumber daya  manusia dan finansial wajib diberikan}. Alokasi sumber daya manusia mengacu kepada bagaimana tugas atau sub-tugas didistribusikan di antara anggota kelompok, sedangkan alokasi finansial mengacu kepada alokasi biaya yang dibutuhkan untuk menjalankan proyek. Alokasi biaya tidak harus berupa biaya yang benar-benar dikeluarkan mahasiswa. Sebagai contoh jika salah satu yang diperlukan dari solusi anda adalah sebuah sensor yang berharga 100.000.000 rupiah dan anda mendapatkan pinjaman sensor tersebut dari salah satu lab di DTETI, maka anda harus memasukkan harga sensor tersebut ke dalam alokasi biaya anda.
}


Pada bagian ini, tuliskan komponen biaya yang mungkin timbul dari pelaksanaan kegiatan \textit{Capstone}. Komponen biaya terdiri dari biaya operasional, seperti pembelian barang/bahan penelitian, biaya pengujian/analisis, penyewaan peralatan, dan penyelenggaraan workshop/pelatihan/\textit{survey}. Komponen biaya mencakup semua biaya yang dikeluarkan oleh semua pihak, baik dari pihak mahasiswa, dosen pembimbing, maupun institusi. Dalam konteks \textit{Capstone}, dimungkinkan komponen biaya bernilai nol, misalnya bila hanya dibutuhkan peralatan berupa laptop yang telah lazim dimiliki setiap individu. Selain rencana anggaran, rencana pelaksanaan kegiatan juga perlu disajikan pada bagian ini. Tuliskan rencana kegiatan mulai dari bulan Agustus 2021 sampai dengan bulan Juli 2022.

\section{Rencana Anggaran Pelaksanaan \textit{Capstone}}

    Bagian ini berisi estimasi biaya yang diperlukan untuk pembelian bahan dan peralatan dalam pelaksanaan kegiatan \textit{Capstone}. Tabel \ref{tab:estimasi_anggaran} merupakan contoh tabel yang berisi estimasi biaya yang diperlukan untuk pelaksanaan kegiatan \textit{Capstone}. Bila bahan atau peralatan sudah tersedia di laboratorium (Lab.), mahasiswa hanya perlu menambahkan penjelasan bahwa bahan dan peralatan sudah tersedia di laboratorium dan tidak perlu menuliskannya di dalam tabel.

        \begin{longtable}{|clcc|c|}
            \caption{Estimasi Anggaran Pelaksanaan Kegiatan \textit{Capstone}}
            \label{tab:estimasi_anggaran}
            \vspace{-0.75em}\\
            \hline
\multicolumn{1}{|c|}{\textbf{No.}} & \multicolumn{1}{c|}{\textbf{Barang}} & \multicolumn{1}{c|}{\textbf{Jumlah}} & \textbf{Harga Satuan} & \textbf{Total Harga} \\ 
            \hline
\multicolumn{1}{|c|}{1}            & \multicolumn{1}{l|}{Lampu LED}       & \multicolumn{1}{c|}{4}               & Rp 47.500,00          & Rp 190.000,00        \\ 
            \hline
\multicolumn{1}{|c|}{2}            & \multicolumn{1}{l|}{Modul Sensor IR} & \multicolumn{1}{c|}{1}               & Rp 28.500,00          & Rp 28.500,00         \\ 
            \hline
\multicolumn{4}{|c|}{\textbf{TOTAL}}                                                                                                     & Rp xxx.xxx,xx        \\ 
            \hline
        \end{longtable}
        
\section{Jadwal Pelaksanaan Kegiatan \textit{Capstone}}

    Berikan \textit{overview} tentang target waktu pelaksanaan proyek akhir ini. Tabel \ref{tab:jadwal_pelaksanaan} di bawah ini merupakan contoh tentang pelaksanaan tugas akhir yang diselesaikan dalam waktu 6 bulan. Pastikan keselarasan jadwal dengan prosedur pelaksanaan. \textbf{Tim mahasiswa wajib menggunakan standard Gantt Chart.}
    
        \begin{longtable}{|l|cccccccccccc|}
            \caption{Jadwal Pelaksanaan Kegiatan \textit{Capstone}}
            \label{tab:jadwal_pelaksanaan}
            \vspace{-0.75em}\\
            \hline
\multicolumn{1}{|c|}{}                                 & \multicolumn{12}{c|}{Bulan ke-}                                                                                                                                                                                                                                                                                                                                                                                                                                                                                                                                          \\ \cline{2-13} 
\multicolumn{1}{|c|}{\multirow{-2}{*}{Tahap Kegiatan}} & \multicolumn{1}{c|}{1}                        & \multicolumn{1}{c|}{2}                        & \multicolumn{1}{c|}{3}                        & \multicolumn{1}{c|}{4}                        & \multicolumn{1}{c|}{5}                        & \multicolumn{1}{c|}{6}                        & \multicolumn{1}{c|}{7}                        & \multicolumn{1}{c|}{8}                        & \multicolumn{1}{c|}{9}                        & \multicolumn{1}{c|}{10}                       & \multicolumn{1}{c|}{11}                       & 12                       \\ \hline
\textbf{Persiapan}                                              & \multicolumn{1}{c|}{}                         & \multicolumn{1}{c|}{}                         & \multicolumn{1}{c|}{}                         & \multicolumn{1}{c|}{}                         & \multicolumn{1}{c|}{}                         & \multicolumn{1}{c|}{}                         & \multicolumn{1}{c|}{}                         & \multicolumn{1}{c|}{}                         & \multicolumn{1}{c|}{}                         & \multicolumn{1}{c|}{}                         & \multicolumn{1}{c|}{}                         &                          \\
a. Studi Literatur                                     & \multicolumn{1}{c|}{\cellcolor[HTML]{333333}} & \multicolumn{1}{c|}{\cellcolor[HTML]{333333}} & \multicolumn{1}{c|}{}                         & \multicolumn{1}{c|}{}                         & \multicolumn{1}{c|}{}                         & \multicolumn{1}{c|}{}                         & \multicolumn{1}{c|}{}                         & \multicolumn{1}{c|}{}                         & \multicolumn{1}{c|}{}                         & \multicolumn{1}{c|}{}                         & \multicolumn{1}{c|}{}                         &                          \\
b. Desain                                              & \multicolumn{1}{c|}{}                         & \multicolumn{1}{c|}{}                         & \multicolumn{1}{c|}{\cellcolor[HTML]{333333}} & \multicolumn{1}{c|}{\cellcolor[HTML]{333333}} & \multicolumn{1}{c|}{}                         & \multicolumn{1}{c|}{}                         & \multicolumn{1}{c|}{}                         & \multicolumn{1}{c|}{}                         & \multicolumn{1}{c|}{}                         & \multicolumn{1}{c|}{}                         & \multicolumn{1}{c|}{}                         &                          \\
c. Pembelian Bahan                                     & \multicolumn{1}{c|}{}                         & \multicolumn{1}{c|}{}                         & \multicolumn{1}{c|}{}                         & \multicolumn{1}{c|}{\cellcolor[HTML]{333333}} & \multicolumn{1}{c|}{\cellcolor[HTML]{333333}} & \multicolumn{1}{c|}{}                         & \multicolumn{1}{c|}{}                         & \multicolumn{1}{c|}{}                         & \multicolumn{1}{c|}{}                         & \multicolumn{1}{c|}{}                         & \multicolumn{1}{c|}{}                         &                          \\
\textbf{Pelaksanaan}                                            & \multicolumn{1}{c|}{}                         & \multicolumn{1}{c|}{}                         & \multicolumn{1}{c|}{}                         & \multicolumn{1}{c|}{}                         & \multicolumn{1}{c|}{}                         & \multicolumn{1}{c|}{}                         & \multicolumn{1}{c|}{}                         & \multicolumn{1}{c|}{}                         & \multicolumn{1}{c|}{}                         & \multicolumn{1}{c|}{}                         & \multicolumn{1}{c|}{}                         &                          \\
a. Pembuatan Prototipe                                 & \multicolumn{1}{c|}{}                         & \multicolumn{1}{c|}{}                         & \multicolumn{1}{c|}{}                         & \multicolumn{1}{c|}{}                         & \multicolumn{1}{c|}{}                         & \multicolumn{1}{c|}{\cellcolor[HTML]{333333}} & \multicolumn{1}{c|}{\cellcolor[HTML]{333333}} & \multicolumn{1}{c|}{}                         & \multicolumn{1}{c|}{}                         & \multicolumn{1}{c|}{}                         & \multicolumn{1}{c|}{}                         &                          \\
b. Pengujian Kinerja                                   & \multicolumn{1}{c|}{}                         & \multicolumn{1}{c|}{}                         & \multicolumn{1}{c|}{}                         & \multicolumn{1}{c|}{}                         & \multicolumn{1}{c|}{}                         & \multicolumn{1}{c|}{}                         & \multicolumn{1}{c|}{\cellcolor[HTML]{333333}} & \multicolumn{1}{c|}{\cellcolor[HTML]{333333}} & \multicolumn{1}{c|}{}                         & \multicolumn{1}{c|}{}                         & \multicolumn{1}{c|}{}                         &                          \\
c. Evaluasi dan Perbaikan                              & \multicolumn{1}{c|}{}                         & \multicolumn{1}{c|}{}                         & \multicolumn{1}{c|}{}                         & \multicolumn{1}{c|}{}                         & \multicolumn{1}{c|}{}                         & \multicolumn{1}{c|}{}                         & \multicolumn{1}{c|}{\cellcolor[HTML]{333333}} & \multicolumn{1}{c|}{\cellcolor[HTML]{333333}} & \multicolumn{1}{c|}{\cellcolor[HTML]{333333}} & \multicolumn{1}{c|}{\cellcolor[HTML]{333333}} & \multicolumn{1}{c|}{\cellcolor[HTML]{333333}} &                          \\
\textbf{Penyelesaian}                                           & \multicolumn{1}{c|}{}                         & \multicolumn{1}{c|}{}                         & \multicolumn{1}{c|}{}                         & \multicolumn{1}{c|}{}                         & \multicolumn{1}{c|}{}                         & \multicolumn{1}{c|}{}                         & \multicolumn{1}{c|}{}                         & \multicolumn{1}{c|}{}                         & \multicolumn{1}{c|}{}                         & \multicolumn{1}{c|}{}                         & \multicolumn{1}{c|}{}                         &                          \\
a. \textit{Finishing}                                           & \multicolumn{1}{c|}{}                         & \multicolumn{1}{c|}{}                         & \multicolumn{1}{c|}{}                         & \multicolumn{1}{c|}{}                         & \multicolumn{1}{c|}{}                         & \multicolumn{1}{c|}{}                         & \multicolumn{1}{c|}{}                         & \multicolumn{1}{c|}{}                         & \multicolumn{1}{c|}{}                         & \multicolumn{1}{c|}{\cellcolor[HTML]{333333}} & \multicolumn{1}{c|}{\cellcolor[HTML]{333333}} &                          \\
b. Pembuatan Laporan                                   & \multicolumn{1}{c|}{}                         & \multicolumn{1}{c|}{}                         & \multicolumn{1}{c|}{}                         & \multicolumn{1}{c|}{}                         & \multicolumn{1}{c|}{}                         & \multicolumn{1}{c|}{}                         & \multicolumn{1}{c|}{}                         & \multicolumn{1}{c|}{}                         & \multicolumn{1}{c|}{}                         & \multicolumn{1}{c|}{}                         & \multicolumn{1}{c|}{\cellcolor[HTML]{333333}} & \cellcolor[HTML]{333333} \\ \hline

        \end{longtable}